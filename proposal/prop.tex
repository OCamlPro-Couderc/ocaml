\section{Semantics in the language}

Namespaces are integrated into the language, they are a meaning to represent
\emph{hierarchies} of modules and namespaces, helping the user to use multiple
modules from different libraries, and possibly with the same name, without using
long-prefixed names or aliases. For the library provider, it allows the
possibility to use simpler names.

Namespaces are not directly used inside the language, but in a prelude of each
module to describe the ``compilation environment''. Only \emph{toplevel modules}
(as compilation unit) can belong to namespaces, not submodules. A namespace is
simply declared: 


%% Namespaces can take arguments, in other words they can abstract some of their
%% implementation and became modular. The argument of the ``functor'' has to be a
%% \texttt{.mli} available in the namespace (in this syntax the module signature is
%% not explicit, it is not clear for now if it shoud be since the mli/cmi is the
%% signature). For now, it is not clear if this argument must be explicit for each
%% modules in the namespace, or only those that uses it. If it has to be explicited
%% (even if unused) in each, the result would be that the interface used to type
%% this argument would also have a ``virtual'' access to it, resulting in a false
%% recursive module.

%% In this proposal, the functorized namespaces are part of the language, not only
%% defined at the build system level (like functor-pack): it is an essential part
%% of our proposal and useful to understand how the namespaces are organized and
%% used.

%% \begin{figure}[H]
\begin{grammar}
<implementation> ::= <header> <structure>

<interface> ::= <header> <signature>

<module_name> = <uident>

<header> ::= <namespace_decl> <imports>
\alt <imports>

<namespace_decl> ::= `in' `namespace' <mod_longident>

<imports> ::= <import> <imports> 

<import> ::= (`with'|`and') <ns_conf> 

<ns_conf> ::= `(' <modules_constraints> `)' `of' <mod_longident>

<modules_constraints> ::= <module_constraint>
\alt <module_constraint> `;' <module_constraints>

<modules_constraint> ::= `_'
\alt <module_name> `as' <module_name>
\alt <module_name> `as' `_'
\alt <module_name>
\end{grammar}
%% \caption{Namespace use}
%% \end{figure}

This construction allows to declare modules into subnamespaces. The following
example is valid:

\begin{OCaml}
in namespace Inria.Std
\end{OCaml}

The syntax is rather simple, and can be simply described. The imports allows the
user to declare which modules to use from this namespace. 

Then an import clause
\begin{OCaml}
with (List) of Inria.Std
\end{OCaml}
can be understood as
\begin{OCaml}
module List = ``module List from namespace Inria.Std''
\end{OCaml}
except this is an alias only on the compiler environment side, i.e. there is no
real module alias added in the code of the program.

The wildcard allows to import in the environment every modules of the namespace
that are not declared in the current constraints. Shadowing doesn't add the
module (it is only useful when using the wildcard), and aliasing allows to have
module from multiple namespace with the same original name. Importing a
namespace does not import the subnamespaces. Being in a namespace allows to
directly import every modules from it. Modules without namespaces are accepted
and virtually doesn't belong to one. Actually, there is a kind of \emph{global}
namespace (like C\# of F\#) for this specific case.

The possibilities inside the language are simple, but there are some possible
extensions that will be discussed later.

\section{Compilation}

\paragraph{Dependencies}

The dependency computation would not change too much and would be helped thanks
to the header. This would allow to retrieve the namespace of each module (except
when using the wildcard import, which is trickier without having knowledge of
the existing namespaces available in the program).

This wildcard issue is only relevant in the case we want namespaces to be given
along with modules' name with the option \texttt{-modules}. It can be solved by
simply changing the behavior of the existing algorithm. For now, asking for raw
dependencies simply prints the modules that are used, without even looking for
the corresponding \texttt{.ml(i)} files in the loadpath. What is proposed is
using the normal behavior and check if a file exists when a namespace cannot be
derived, checking if files that corresponds uses one namespace from the imports
(that only use wildcards). It can refine the dependencies of the compiled
program, and let the build system deal with modules whose namespaces aren't
found (because they belong to external libraries that are linked with the
program). This algorithm would be used however in the original ocamldep's
behavior, when looking for the .ml in the loadpath: when namespace cannot be
resolved, each file in the loadpath with the same name as the module is checked
to possibly retrieve one that belongs to a namespace imported with wildcard. It
can be an option of ocamldep, and it will simply read the first tokens of the
files that could match to retreive the namespace.


\paragraph{Compilation unit}

Since namespaces allows the possibility of modules with the same name to be
linked together, internal name must be extended to be a long name, which is
simply using the namespace as a prefix. The prefix can be the namespace and
maybe some other information, like a version information: it would allow to link
two versions of the same library in the same program. A use case of such a
feature could be for debugging features that have been rewritten between two
versions, or simply benchmarking a new version against the old one. This also
justifies the need to extend the compiler instead of making a syntax extension.


\paragraph{Linking} We will suppose for now that linking is simply giving a name
of a library (or a path), without bothering for looking how are organized those
libraries on the filesystem. It will be explained in the next section.

An important property of namespaces is they are \emph{extensible}, meaning that
anyone can add a module to an existing namespace. In other words, linking two
libraries that exports the same namespaces will \emph{merge} them into
one. Actually, it is the current behavior during link with modules: if two
modules have the same name in the loadpath, only the first one found (according
to the order of -I given to the compiler) is used (see formula (\ref{merge}). In
our merge semantics, we follow the same rules: when two namespaces have the same
name, the result is a namespace that contains the modules from both, and in case
of conflicting modules only the first one found are kept.

A formal representation is the following: suppose the loadpath $LP$, which
contains paths $P_i$. Each path $P_i$ is a directory that contains compilation
units $C_j$ and directories $D_k$. Then, the compilation units are:

\begin{multline}
\forall D_1, D_2 \in P_1, P_2, D_1 = D_2
\rightarrow P_1 \oplus' P_2 = P_1 \oplus P_2 + D_1 \oplus D_2
\label{ns-merge} 
\end{multline}

It is a recursive definition, meaning that subfolders are also taking in
consideration. As a result $\oplus$ is replaced by $\oplus'$ when computing the
available compilation units.

For example, we can consider the two following libraries:

\begin{verbatim}
(* "global" is the root namespace *)

(* For stdlib *)
global ---- Std ---- List
                  |- Hashtbl
                  |- ...

(* For ocamlpro *)
global ---- Std ---- List
         |- ...   |- Utils
                  |- ...
\end{verbatim}
and we suppose the library \emph{ocamlpro} is linked using \texttt{-I +ocamlpro}.

\texttt{stdlib} is always linked last, meaning that any libraries using the
\texttt{``global''.Std} namespace would conflict with it. The result is that the
modules in \texttt{Ocamlpro/Std} would subsume those existing with the same name
from the standard library. If one use \texttt{List} in its program, it would be
the one from OCamlPro. But modules that do not not exists in
\texttt{Ocamlpro/Std} are still accessible.


\paragraph{Parametric namespaces}

Namespaces can be parametric and use multiple implementations over one
interface. A module can be abstracted and considered as a \emph{parameter} in
the namespace. 

The actual functor pack patch adds two fields to cmis/cmos/cmxs:
\lstinline{functor_args} and \lstinline{functor_parts}: the first one declares
the abstracted interfaces, while the second contains the first and the
dependencies that are also functorized. It will be completely explicit in the
build system. For example, let assume a namespace \texttt{Ns} which is
parametric over an interface \texttt{Arg}. Its modules are compiled with the
option

\begin{verbatim}
ocamlc -functor arg.cmi a.ml
\end{verbatim}

We can then instantiate them by applying an argument, which would result on
a new namespace that corresponds of the applied version on this specific
argument. Instantiating the modules would lead to 

\begin{verbatim}
ocamlc -apply x.cmo a.cmo 
\end{verbatim}

As a result, the instantiation is done once and for all. This would introduce a
concept of ``functor-units''. The namespace of the applied version of the
functor will be the same as the argument. It can be rather restrictive, but it
is easy to justify: when using a module to apply the functor, this module can be
rarely coerced into the correct argument. As a result, it must be rebinded into
another compilation unit to match the correct fields, and then it can belong to
the correct namespace.

\section{Installation and file system relationship}

One of the advantage of namespaces is their ability to provide a nice way to
define how libraries are installed and how the compiler should look for the
modules. As a result, we can leverage the need for external tools to compile
using external libraries, especially for simple programs.

\paragraph{Compiled modules and installation}

Namespaces and file system organization are highly related: a namespace is a
directory, at least after compilation (it has been hinted in (\ref{ns-merge})
actually). For example, if the module belongs to the namespace
\texttt{Inria.Std}, its compilation result will be placed in the directory
\texttt{inria/std}. As a result, the compiler will be able to look for modules
in namespaces and subnamespaces simply by scanning recursively the folders in
its loadpath. Sources should belong to the same directory, and it could help
ocamldep to look for dependencies without having to check the ``in namespace''
declaration. However, it is not mandatory for the sources to be put in the
correct folder, but since the compiler cannot create directories it might be
useful. Moreover, it is the only possibility to distinguish files with the same
name.

When linking, those folders are looked for from the loadpath: there is not a
global folder where modules from different libraries are placed. It would be
quite hard to link a library since the compiler would have to look for each
modules that belongs to that library. Moreover, installing a library that
extends a namespace and redefines some of its existing modules would be clearly
dangerous, and could lead to overwriting files.

To allow those namespaces to be used without external tool, we propose to add a
canonical path where libraries should be installed, which would be symbolized by
``+'' before the folder name with \texttt{-lib}. This path can be set by an
environment variable of the system (\texttt{OCAML\_LD\_PATH} for
example). Linking a library is simply giving the folder where it is installed to
\texttt{-lib}. Modules that doesn't have a namespace can be found at the root of
the loadpath. This is not implemented in the current prototype.

Having such a standard library path for user libraries is relevant, thanks to
the use of OPAM: the user's libraries are installed in the
\texttt{~/.opam/<version>/lib/} directory, and it can be used as a standard
installation path. 

\paragraph{Alternative to hierarchical filesystem}

Whereas file system is a good way to organize namespaces, some could want a way
to flatten this hierarchy to avoid scanning (it can be too slow on some
operating system). The idea is to provide a file (\texttt{.ns}) that simply
describes where are the files of the library, and in which namespace they
belong. It has the following simple syntax:

\begin{grammar}
<lines> ::= (* empty *)
<line> `\\n' <lines>

<line> ::= <filepath> `:' <mod_longident>
\end{grammar}

The compiler can create this file during compilation, or we can imagine a simple
program (\emph{ocamlns}) which would read the namespace information of each file
to create this \texttt{.ns}. It would be available at the root of each installed
libraries (in each -lib added folder) and read by the compiler.


\section{Others concerns and future extensions}

\paragraph{-pack with namespace}

The problem namespaces resolves is the -pack issue, however some would still
want to use both in a program. In that case, the packing result maybe could have
a namespace. The behavior is then:
\begin{itemize}
\item if every modules to be packed are in the same namespace, then the
  ``package'' is in the same.
\item if namespaces are different, -pack fails.
\end{itemize}

In this constraints, the -pack is backward compatible since modules without
namespaces doesn't belong to a namespace at all.

\paragraph{-functor-pack with namespaces}

The rules are the same as with -pack: the \emph{big functor} belongs to the same
namespace as its functors packed.

\paragraph{Automatically opened modules}

Namespaces can be an elegant way to declare libraries. We can suppose a library
wants one of its modules is always opened when using it, ``à-la''
Pervasives. A possibility would be to open automatically modules whose name is
Pervasives when using a namespace. It binds the name to a specific semantics,
but is relevant since some languages binds the \emph{main} name as the entry
point of a program.

\section{Use-case}


\subsection{Extending libraries}

Namespaces are extensible, in other words, one can add any module in it. It can
be useful to extend the standard library, or replacing some modules with an
implementation more efficient. We suppose the standard OCaml library belongs to
the namespace Inria.Std (we suppose its modules are automatically
imported).

\begin{OCaml}
(** hashset.ml *)
in namespace Inria.Std
...
\end{OCaml}

\begin{OCaml}
(** string.ml, a replacement for the original String module *)
in namespace Inria.Std
...
\end{OCaml}

Supposing this extended library is installed into \texttt{+extendedlib}, the
compilation of a program M using it is simply:
\begin{verbatim}
ocaml(c|opt) -o prog -lib +extendedlib m.ml
\end{verbatim}

Of course, the String module we've defined is not used by the modules from the
``original'' namespace but only by our extended version.

\subsection{Parameterized library}

Suppose we're writing a library that uses a specific asynchronous monad, like
Lwt, but without without specifying implementation, leaving the choice up to the
user. Suppose this library is an implementation for the HTTP protocol. The modules
from this library resides in the namespace Mylib.Http. Protocol is the main module.

\begin{OCaml}
(** async_monad.mli *)
in namespace Mylib.Http

type t

val (>>=) : 'a t -> ('a -> 'b t) -> 'b t

val return : 'a -> 'a t
val bind : 'a t -> ('a -> 'b t) -> 'b t (* == (>>=) *)
val poll : 'a t -> 'a option
...
\end{OCaml}

\medskip

\begin{OCaml}
(** protocol.ml *)

in namespace MyLib.Http

open Async_monad
...
\end{OCaml}

The compilation of our module is:

\begin{verbatim}
ocamlc -c -functor async_monad.cmi protocol.ml
\end{verbatim}

\medskip

Now suppose your writing a server, using this HTTP library (we suppose it
belongs to a namespace Myserver). First of all, we have to instantiate our
namespace over an implementation of our monad:

\begin{OCaml}
in namespace Mylib.Http_Lwt
with (Lwt) of Ocsigen

include Lwt
\end{OCaml}

\begin{verbatim}
ocamlc -apply mylib/http_Lwt/lwt.cmo protocol.cmo
... # We do the same for the other modules in this namespace 
\end{verbatim}

We then create a new namespace Mylib.Http\_Lwt that corresponds to the
instantiation of Mylib.Http over Ocsigen.Lwt.

\begin{OCaml}
(** main_lwt.ml *)
in namespace Myserver
  with (Protocol; Lwt) of MyLib.Http_Lwt
...
\end{OCaml}

\subsection{Multiple instantiation}

Suppose a library using a SAT-solver to prove the correctness of some values it
is using. There exists multiple SATisfiability solving algorithms (mainly DPLL
and CDCL, with multiple variants), and one case of functors at the namespace
level would be to instantiate multiple versions of a library and benchmark the
different implementations.

Suppose our library, particularly composed by a \emph{hole} Sat and a module
Solver. The library defines its namespace MySolver.

\begin{OCaml}
(** sat.mli *)
in namespace MySolver

type bnf

val solve: bnf -> bool
\end{OCaml}

\medskip 

\begin{OCaml}
(** solver.ml *)
in namespace MySolver

let is_sat f =
... 
(* some preliminary computations, like turning the formula into its bnf *)
  let sat = Sat.solve formula in 
...
\end{OCaml}

\begin{verbatim}
ocamlc -c sat.mli
ocamlc -c -functor sat.cmi solver.ml
\end{verbatim}

\medskip

Finally, our Solver can be used by the following benchmarking modules:

\begin{verbatim}
ocamlc -apply mySolver_Dpll/dpll.cmo solver.cmo
... # every other modules to instantiate

ocamlc -apply mySolver_Cdcl/cdcl.cmo solver.cmo
... # idem 
\end{verbatim}

\begin{OCaml}
(** bench.ml *)
with (Parser, Solver as Solve_Dpll) of MySolver_Dpll
and (Solver as Solve_Cdcl) of MySolver_Cdcl

...
let compute_and_print_time l = ...
...

let () =
  let formulas = 
    Parser.open_read_and_parse Sys.argv.[1] in
  let eval f =
    let t1 = Sys.time () in
    ignore (Solve_Dpll.is_sat f);
    let t2 = Sys.time () in
    ignore (Solve_Cdcl.is_sat f);
    let t3 = Sys.time () in
    (t2 - t1, t3 - t2)
  in
  let times = List.map eval formulas in
  compute_and_print_time times
\end{OCaml}

\subsection{Importing modules with the same name}

One problem that can be easily solved is the ability to use two compilation
units that share the same name, thanks to the importing constraints and even the
full qualified path of the second proposal. Both proposal can use constraints,
but second can be even more powerful.

For example, suppose the Batteries modules are not prefixed with
\texttt{``Bat\_''}, it would not be possible to use List from the Stdlib against
the one from Batteries.

\begin{OCaml}
with (List as BList) of Batteries
and (List as OList) of Inria.Std
\end{OCaml}

With the second proposal, aliasing is possible but not mandatory since it makes
qualification explicit.

\begin{OCaml}
with Batteries.(List)
and Inria.Std.(List)
\end{OCaml}
